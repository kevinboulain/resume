% -- modes ---------------------------------------------------------------------
\definemode[anonymous][keep] % Some recruiters like an anonymized resume.

\enabletrackers[ % \showtrackers
  % context.trace, % What is passed to TeX.
  % resolvers.readfile, % File errors, mainly system-related?
  % typesetters.suspects, % https://wiki.contextgarden.net/Trackers/typesetters.suspects
  % visualizers.justification, % https://wiki.contextgarden.net/Trackers/visualizers.justification
]
\enabledirectives[ % \showdirectives
  logs.errors,
  % Link debugging https://www.mail-archive.com/ntg-context@ntg.nl/msg73866.html
  % destinations.log,
  % references.border,
]

% -- colors --------------------------------------------------------------------
% https://material.io/guidelines/style/color.html
\definecolor[material_blue_gray_500][h=607d8b]
\definecolor[material_blue_gray_700][h=455a64]
\definecolor[material_blue_gray_900][h=263238]

% -- contact information -------------------------------------------------------
\doifmodeelse{anonymous}{
  \edef\name{Kevin B.}
}{
  \edef\name{Kevin Boulain}
  \edef\location{France}
  \edef\mobile{+33 6 95 49 89 35}
  \edef\email{kevin% Some...
              @% ...obfuscation.
              boula.in}
}

% -- document setup ------------------------------------------------------------
\setuppapersize[A4][A4] % default
\setupbodyfont[10pt]
\setuplayout[ % http://wiki.contextgarden.net/Layout
  backspace=30mm,
  footer=0mm,
  header=0mm,
]

\definecolor[hyperlink][material_blue_gray_700]
\setupinteraction[ % http://wiki.contextgarden.net/Interaction
  state=start,
  style=, % Don't default to bold.
  color=hyperlink, % Destination is not on the same page.
  contrastcolor=hyperlink, % Destination is on the same page.
  focus=standard, % Don't change the zoom.
  title={\name's resume},
  author={\name},
]

% -- content style -------------------------------------------------------------
\setupwhitespace[halfline]

\definehead[contact][title]
\setuphead[contact][
  after=\nowhitespace, % No space after title.
  color=material_blue_gray_900,
]

\doifmodeelse{anonymous}{
  \definebuffer[contactinfos] % Define the item group as a buffer so it gobbles everything...
}{
  \definestartstop[contactinfos] % ...otherwise just use a dummy environment.
}

\setupmarginframed[left][
  align={flushright,broad}, % Actually not necessary https://www.mail-archive.com/ntg-context@ntg.nl/msg88886.html
]
\def\experiencecategory#1{\inleft{\color[material_blue_gray_900]{#1}}}

\setuptextrules[
  color=material_blue_gray_700,
  rulecolor=material_blue_gray_500,
]
\def\experiencetitle#1{\textrule[top]{#1}}

\def\experiencelocality#1#2{{\em#1\hfill#2}}

\defineitemgroup[experienceactivity]
\setupitemgroup[experienceactivity][each][
  2, % Hyphen.
  joinedup, % No space before and after.
  packed, % No space between items.
]

% -- content -------------------------------------------------------------------
\starttext

  \contact{\name} % ------------------------------------------------------------
  \startcontactinfos
    \startlines
      \location
      \mobile
      \goto{\email}[url(mailto:\email)] % Goto is necessary here.
    \stoplines
  \stopcontactinfos
  \blank[3*big]

  \experiencecategory{Experience} % --------------------------------------------

  \useURL[kadiska] % -----------------------------------------------------------
    [https://kadiska.com/]
    []
    [Netskope / Kadiska]

  \experiencetitle{Staff Engineer}
  \experiencelocality
    {\from[kadiska], Remote, France}
    {November 2023 – Present} % 2023-11-13

  Digital experience monitoring.

  \startexperienceactivity
  \stopexperienceactivity

  \useURL[google] % ------------------------------------------------------------
    [https://www.google.com/]
    []
    [Google]

  \useURL[cloudarmor]
    [https://cloud.google.com/armor]
    []
    [Cloud Armor]

  \experiencetitle{Site Reliability Engineer}
  \experiencelocality
    {\from[google], Zürich, Switzerland}
    {April 2019 – February 2023} % 2019-04-01 / 2023-02-28

  DDoS protection infrastructure and \from[cloudarmor].

  \startexperienceactivity
    \item Design, implementation, and migration of multi-million queries per
          second systems to a new infrastructure platform.
    \item Overhaul of the legacy rollout system, monitoring, alerting,
          introduction of automated qualification, capacity planning, probers,
          CI, ...
    \item Code mostly written in Python.
    \item On-call.
  \stopexperienceactivity

  \useURL[performancevision] % -------------------------------------------------
    [https://www.performancevision.com/]
    []
    [Accedian / Performance Vision]

  \useURL[drda]
    [https://pubs.opengroup.org/onlinepubs/9690989699/toc.pdf]
    []
    [DRDA]

  \useURL[tls]
    [https://www.rfc-editor.org/rfc/rfc5246]
    []
    [TLS]

  \useURL[msrpc]
    [https://winprotocoldoc.blob.core.windows.net/productionwindowsarchives/MS-RPCE/\%5BMS-RPCE\%5D.pdf]
    []
    [MSRPC]

  \useURL[vxlan]
    [https://www.rfc-editor.org/rfc/rfc7348]
    []
    [VXLAN]

  \useURL[ansible]
    [https://www.ansible.com/]
    []
    [Ansible]

  \useURL[lxc]
    [https://linuxcontainers.org/]
    []
    [LXC]

  \useURL[qemu]
    [https://www.qemu.org/]
    []
    [QEMU]

  \useURL[kvm]
    [https://www.linux-kvm.org/page/Main_Page]
    []
    [KVM]

  \experiencetitle{Software Engineer}
  \experiencelocality
    {\from[performancevision], Paris, France}
    {December 2014 – March 2019} % 2014-12-01 / 2019-03-19

  Network performance monitoring appliance.

  \startexperienceactivity
    \item Extensions to a real-time sniffer written in C, including protocol
          parsers such as \from[drda], \from[tls], \from[msrpc], \from[vxlan],
          ..., and data analysis.
    \item Design, implementation, migration, and maintenance of the appliance's
          operating system. Complete overhaul of the build and installation
          process for virtual and physical machines.
    \item Contributions to the Python backend.
    \item Complete overhaul of the CI.
    \item Overhaul and administration of the infrastructure (\from[ansible],
          \from[lxc], \from[qemu]/\from[kvm]).
    \item Support.
  \stopexperienceactivity

  \useURL[nbs] % ---------------------------------------------------------------
    [https://www.nbs-system.com/]
    []
    [NBS System]

  \useURL[xen]
    [https://xenproject.org/]
    []
    [Xen]

  \useURL[salt]
    [https://saltproject.io/]
    []
    [Salt]

  \experiencetitle{DevOps}
  \experiencelocality
    {\from[nbs], Paris, France}
    {June 2014 - September 2014} % 2014-06-02 / 2014-09-05

  E-commerce managed hosting.

  \startexperienceactivity
    \item Tooling development for administrating the infrastructure (written in
          Python & \from[salt]), including reworking the deployment of
          \from[xen] virtual machines.
    \item Realization of proofs of concepts for new services.
  \stopexperienceactivity

  \useURL[42] % ----------------------------------------------------------------
    [http://www.42.fr/]
    []
    [School 42]

  \useURL[vmware]
    [https://www.vmware.com/]
    []
    [VMware]

  \useURL[openldap]
    [https://www.openldap.org/]
    []
    [OpenLDAP]

  \useURL[freeradius]
    [https://freeradius.org/]
    []
    [FreeRADIUS]

  \useURL[kerberos]
    [https://web.mit.edu/kerberos/]
    []
    [Kerberos]

  \useURL[ZFS]
    [https://en.wikipedia.org/wiki/ZFS]
    []
    [ZFS]

  \experiencetitle{Technical and Educational Responsible}
  \experiencelocality
    {\from[42], Paris, France}
    {May 2013 - May 2014} % 2013-05-02 / 2014-05-31

  IT school.

  \startexperienceactivity
    \item Design, deployment, and administration of the infrastructure
          (\from[vmware] servers, a thousand of macOS clients, thousands of new
          student accounts each year, custom solutions built around
          \from[openldap], \from[freeradius], \from[kerberos], \from[ZFS], ...).
    \item Development of a centralized administrative tool written in Python.
    \item Support.
    \item Educational role for students learning the C language through projects
          and e-learning.
  \stopexperienceactivity

  \useURL[ionis] % -------------------------------------------------------------
    [http://www.ionis-group.com/]
    []
    [IONIS Education Group]

  \experiencetitle{System and Network Administrator}
  \experiencelocality
    {\from[ionis], Kremlin-Bicêtre, France}
    {July 2011 – April 2013} % 2011-07-01 / 2013-04-30

  Technology schools.

  \startexperienceactivity
    \item Administration of the infrastructure (thousands of client machines,
          fifteen thousand accounts for students and staff).
    \item Migration of the legacy infrastructure to \from[vmware].
    \item Support.
    \item Educational role for students learning the C language through projects
          and e-learning.
  \stopexperienceactivity

  \page[no] % Depending on previous experiences, a page break might look better.
  \experiencecategory{Education} % ---------------------------------------------

  \useURL[epitech] % -----------------------------------------------------------
    [http://www.epitech.eu/]
    []
    [EPITECH]

  \experiencetitle{Preparation for the Master's Degree}
  \experiencelocality
    {\from[epitech], Kremlin-Bicêtre, France}
    {October 2010 – March 2014} % 2010-10-04 / 2014-03-28

  Studies leading to a Master’s degree at the European Institute of Technology
  stopped at the end of the 4\high{th} year.

  \useURL[branly] % ------------------------------------------------------------
    [https://www.lyceebranlydreux.com/]
    []
    [Édouard Branly high shcool]

  \experiencetitle{Baccalauréat}
  \experiencelocality
    {\from[branly], Dreux, France}
    {2010}

  Baccalauréat in the engineering sciences.

\stoptext
